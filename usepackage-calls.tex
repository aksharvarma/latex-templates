
% The standard way to use the beamer class
%
% dvipsnames is needed to get nice names for colors
% aspectratio makes it more suited to the 16:9 ratio on my laptop
%
% \documentclass[aspectratio=169,dvipsnames,handout]{beamer}
%
%

% For my math stuff
\usepackage{math-commands-preamble}
%
% Including math-commands-preamble already includes:
%
% \usepackage{amsmath}        % Math typesetting
% \usepackage{amsthm}         % Theorem and other such environments for math
% \usepackage{amsfonts}       % Fonts specific to math
% \usepackage{amssymb}        % Math symbols
% \usepackage{mathtools}      % more math related tools
% \usepackage{bm}             % For better bold symbols in math land
% \usepackage{latexsym}       % LaTeX symbols font
% \usepackage{nicefrac}       % For better fractions in text: 1/2 will be better.
% \usepackage{mathdots}       % Improves \ddots, \vdots, and provides \iddots
%%%% %%%% %%%% %%%% %%%% %%%% %%%% %%%%
%%%% %%%% %%%% %%%% %%%% %%%% %%%% %%%%


% Geometry
\usepackage[left=0.1\paperwidth, right=0.1\paperwidth, textwidth=0.8\paperwidth, top=0.1\paperheight, bottom=0.1\paperheight, textheight=0.8\paperheight, marginparwidth=0.15\paperwidth, marginparsep=3ex]{geometry}
% Also set all indentation to 0.
% \setlength\parindent{0pt}
% \usepackage[indent=0pt, parfill=3em]{parskip}
\usepackage{parskip}            %Handle paragraph indentation a little better.


\usepackage[utf8]{inputenc}            % Input encoding to allow utf-8 input
\usepackage[T1]{fontenc}               % use 8-bit T1 fonts
\usepackage{microtype}                 % microtypography
\usepackage[dvipsnames]{xcolor}        % See also: https://en.wikibooks.org/wiki/LaTeX/Colors
\usepackage{graphicx}                  % For including images in the PDF.
\graphicspath{{./graphs/},{./images/}} %folders to check for images/graphs in.
\usepackage{float}                     % nicer tables/figures/other floating boxes
\usepackage{wrapfig}                   % wrap text content all around figures

% Table specific
\usepackage{booktabs}           % For professional quality tables
\usepackage{array}              % For better tables
\usepackage{multirow}           % For fancy stuff in tables
\usepackage{multicol}           % For dealing with content on multiple columns
\usepackage{caption}            % ?????
\usepackage{subcaptions}        % for captioning and labeling subfigures
\newcommand{\PreserveBackslash}[1]{\let\temp=\\#1\let\\=\temp}
\newcolumntype{C}[1]{>{\PreserveBackslash\centering}m{#1}}
\newcolumntype{R}[1]{>{\PreserveBackslash\raggedleft}m{#1}}
\newcolumntype{L}[1]{>{\PreserveBackslash\raggedright}m{#1}}



\usepackage{enumitem}           % Better enumerate environment
\usepackage{tikz}               % For any drawings done in LaTeX itself.
\usepackage{etoolbox}           % if/then/else, booleans, and other tools and goodies
\usepackage{fancyhdr}           % Page headers for exams, details are available below.


% This requires no packages at all
\newcommand{\graffiti}[1]{\marginpar{\textit{#1}}}
% If you want fancier:
\usepackage{marginnote}         % margin graffiti (for side comments in solutions).
\newcommand{\graffiti}[1]{\marginnote{\raggedright\textsl{#1}}}
% This command is all that is needed, so putting it here instead of with others.


\usepackage{algorithm}
\usepackage{algpseudocode}
\usepackage{listings}

\usepackage{url}                % simple URL typesetting
\usepackage{hyperref}           % Handling hyperlinks
\hypersetup{                    % some setup for URLs/hyperlinks
    colorlinks,
    linkcolor={red!50!black},
    citecolor={blue!50!black},
    urlcolor={blue!80!black}
}

% AFTER hyperref
\usepackage[capitalise,noabbrev]{cleveref}           % For handling references better.



% Unsure if needed but might be useful when adding PDFs/tikz figures as is.
\usepackage{standalone}
%
% All require packages should end above here. Other than pgfkeys and pgfopts
%


% I think this was best done in the repo for course problem sets.
% The following is for setting options for this package
% Primarily used for author/course name, settings type of handout and so on.
%
% kvoptions-patch was suggested to be added before pgfkeys
% so that spaces and commas in values are handled better.
% \usepackage{kvoptions-patch}
% A quick overview and some simple examples for pgfkeys can be found here:
% http://www.tug.org/TUGboat/tb30-1/tb94wright-keyval.pdf
\usepackage{pgfkeys}
% Similarly, for pgfopts, the manual is a good thing to read:
% http://mirrors.ctan.org/macros/latex/contrib/pgfopts/pgfopts.pdf
\usepackage{pgfopts}
% Another simple example pgfkey setup (beyond what's there in the Wright PDF)
% is available at: https://tex.stackexchange.com/a/34318
% The full documentaton is a section in the PGF manual:
% http://mirrors.ctan.org/graphics/pgf/base/doc/pgfmanual.pdf


% Various font related things
\usepackage{concrete-math-fonts}
\usepackage{lmodern}
\usepackage{helvet}
\usepackage{fourier}            % Fits well with the Euler math stuff
\usepackage[utopia]{mathdesign}
\usepackage[charter]{mathdesign}
\usepackage{inconsolata}
\usepackage{montserrat}
